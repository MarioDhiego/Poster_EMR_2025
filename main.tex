%----------------------------------------------------------------------------------------
%	CONFIGURACOES
%----------------------------------------------------------------------------------------
\documentclass[a0,portrait]{a0poster}
\setlength{\paperwidth}{90cm}
\setlength{\paperheight}{120cm}

\usepackage[top=3cm, bottom=0.3cm, left=4cm, right=6cm]{geometry}
\usepackage[compact]{titlesec}
\usepackage{lipsum}

\usepackage{multicol}
\columnsep=100pt

\usepackage[usenames,dvipsnames,svgnames,table]{xcolor}
\definecolor{gray}{rgb}{0.7,0.7,0.7}
\definecolor{vinho}{HTML}{A50021}
\definecolor{rosaFundo}{HTML}{F2DADA}
\definecolor{cinzaFundo}{HTML}{BFBFBF}
\definecolor{cinzaTabela1}{HTML}{E6E6E6}
\definecolor{cinzaTabela2}{HTML}{CACACA}
\usepackage{sectsty}
\sectionfont{\color{vinho}}

\usepackage{fancybox}
\usepackage{graphicx}
\graphicspath{{figures/}}

\usepackage{tikz}
\usetikzlibrary{shadows,calc}
% some parameters for customization
\def\shadowshift{2pt,-2pt}
\def\shadowradius{4pt}

\colorlet{innercolor}{black!60}
\colorlet{outercolor}{gray!05}

% this draws a shadow under a rectangle node
\newcommand\drawshadow[1]{
    \begin{pgfonlayer}{shadow}
        \shade[outercolor,inner color=innercolor,outer color=outercolor] ($(#1.south west)+(\shadowshift)+(\shadowradius/2,\shadowradius/2)$) circle (\shadowradius);
        \shade[outercolor,inner color=innercolor,outer color=outercolor] ($(#1.north west)+(\shadowshift)+(\shadowradius/2,-\shadowradius/2)$) circle (\shadowradius);
        \shade[outercolor,inner color=innercolor,outer color=outercolor] ($(#1.south east)+(\shadowshift)+(-\shadowradius/2,\shadowradius/2)$) circle (\shadowradius);
        \shade[outercolor,inner color=innercolor,outer color=outercolor] ($(#1.north east)+(\shadowshift)+(-\shadowradius/2,-\shadowradius/2)$) circle (\shadowradius);
        \shade[top color=innercolor,bottom color=outercolor] ($(#1.south west)+(\shadowshift)+(\shadowradius/2,-\shadowradius/2)$) rectangle ($(#1.south east)+(\shadowshift)+(-\shadowradius/2,\shadowradius/2)$);
        \shade[left color=innercolor,right color=outercolor] ($(#1.south east)+(\shadowshift)+(-\shadowradius/2,\shadowradius/2)$) rectangle ($(#1.north east)+(\shadowshift)+(\shadowradius/2,-\shadowradius/2)$);
        \shade[bottom color=innercolor,top color=outercolor] ($(#1.north west)+(\shadowshift)+(\shadowradius/2,-\shadowradius/2)$) rectangle ($(#1.north east)+(\shadowshift)+(-\shadowradius/2,\shadowradius/2)$);
        \shade[outercolor,right color=innercolor,left color=outercolor] ($(#1.south west)+(\shadowshift)+(-\shadowradius/2,\shadowradius/2)$) rectangle ($(#1.north west)+(\shadowshift)+(\shadowradius/2,-\shadowradius/2)$);
        \filldraw ($(#1.south west)+(\shadowshift)+(\shadowradius/2,\shadowradius/2)$) rectangle ($(#1.north east)+(\shadowshift)-(\shadowradius/2,\shadowradius/2)$);
    \end{pgfonlayer}
}

% create a shadow layer, so that we don't need to worry about overdrawing other things
\pgfdeclarelayer{shadow} 
\pgfsetlayers{shadow,main}

\newsavebox\mybox
\newlength\mylen

\newcommand\shadowimage[2][]{%
\setbox0=\hbox{\includegraphics[#1]{#2}}
\setlength\mylen{\wd0}
\ifnum\mylen<\ht0
\setlength\mylen{\ht0}
\fi
\divide \mylen by 120
\def\shadowshift{\mylen,-\mylen}
\def\shadowradius{\the\dimexpr\mylen+\mylen+\mylen\relax}
\begin{tikzpicture}
\node[anchor=south west,inner sep=0] (image) at (0,0) {\includegraphics[#1]{#2}};
\drawshadow{image}
\end{tikzpicture}}

\usepackage{caption}
\usepackage{amsfonts, amsmath, amsthm, amssymb}
\usepackage{wrapfig}
\usepackage[portuges]{babel}
% \usepackage[utf8]{inputenc}

\usepackage[alf]{abntex2cite}

\usepackage{helvet}
\renewcommand{\familydefault}{\sfdefault}
\usepackage{fontspec}
\newfontfamily\boogie{RugeBoogie}[Path=fonts/, UprightFont=*-Regular]

\usepackage{etoolbox}
\AtBeginEnvironment{thebibliography}{\normalsize}
\usepackage[font=normalsize,labelfont=bf]{caption}
\usepackage{mwe}

%----------------------------------------------------------------------------------------
%	CABEÇALHO
%----------------------------------------------------------------------------------------
\begin{document}

\begin{minipage}[c]{\linewidth}
	\vspace{0.1cm}
	
    \noindent\makebox[\textwidth][c]
    {
    	% Imagem a esquerda
		\begin{minipage}[c]{0.10\textwidth}
        	\vspace{0pt}\raggedright
            \hspace{1cm}
            \includegraphics[width=\linewidth]{example-image-a}
        	\includegraphics[width=\linewidth]{example-image-b}
		\end{minipage}

        \hfill

		% Titulo e Autores
        \fcolorbox{rosaFundo}{rosaFundo}{
        \begin{minipage}[c]{0.70\textwidth}
        	\centering
        	\vspace{1cm}
                \Huge \textbf{Comparação e Desempenho dos Testes de Normalidade Implementados na Linguagem R via Simulação de Monte Carlo}\\[0.5cm]
        	\LARGE \textbf{Autor$^{1}$, Autor$^{2}$}\\[0.2cm]
        	\large 1. Departamento, Universidade\\
                \large 2. Departamento, Universidade\\[0.1cm]
        	\small \texttt{Email: autor@email.br, autor@email.br}\\
            \vspace{1cm}
        \end{minipage}}

        % Imagem a direita
        \begin{minipage}[c]{0.15\textwidth}
        \vspace{0pt}\raggedleft
        \includegraphics[scale=0.5,width=\linewidth]{example-image-c}
        \hspace{1cm}
        \end{minipage}
  }
  \noindent\makebox[\linewidth]{
  	\textcolor{vinho}{
    	\rule{\paperwidth}{5pt}
        }
	}
\\[0.1cm]

%----------------------------------------------------------------------------------------
\vspace{0.5cm}
\begin{multicols}{2}
%----------------------------------------------------------------------------------------
%	INTRODUÇÃO
%----------------------------------------------------------------------------------------
\section{Introdução}

\lipsum[1-2]

%----------------------------------------------------------------------------------------
%	OBJETIVO
%---------------------------------------------------------------------------------------- 
\section{Objetivo ou Proposta}

\lipsum[3-4]

\lipsum[5]

\begin{wrapfigure}[13]{L}{.35\linewidth}
\noindent\shadowimage[width=\linewidth]{example-image-plain}
\captionof{figure}{Legenda}
\end{wrapfigure}

\lipsum[6-7]

%----------------------------------------------------------------------------------------
%	FUNDAMENTAÇÃO TEÓRICA
%---------------------------------------------------------------------------------------- 
\section{Fundamentação Teórica}

\lipsum[8-10]

%----------------------------------------------------------------------------------------
%	METODOLOGIA
%----------------------------------------------------------------------------------------
\section{Metodologia}

\lipsum[11-12]

{
\centering
\noindent\shadowimage[width=.55\linewidth]{example-image-plain}
\captionof{figure}{Legenda}
\label{fig:rede}
}

\lipsum[13]

%----------------------------------------------------------------------------------------
%	RESULTADOS
%----------------------------------------------------------------------------------------
\section{Resultados teóricos/prático e/ou análises (se houver)}

\lipsum[14-15]

\vphantom{1}\\
{
\centering
\begin{tikzpicture}
\node[anchor=south west,inner sep=0] (table) at (0,0)
{\rowcolors{2}{cinzaTabela1}{cinzaTabela2}
   \begin{tabular}{|c|c|c|c|c|}
       \hline
       \rowcolor{gray}
       Cabeçalho 1 & Cabeçalho 2 & Cabeçalho 3 & Cabeçalho 4 & Cabeçalho 5 \\ \hline
       Valor 11    & Valor 12    & Valor 13    & Valor 14    & Valor 15    \\ \hline
       Valor 21    & Valor 22    & Valor 23    & Valor 24    & Valor 25    \\ \hline
   \end{tabular}
};
\drawshadow{table}
\end{tikzpicture}
\captionof{figure}{Resultados dos testes de tradução em BLEU \textit{score}}
\label{fig:resultados}
}
\vphantom{1}\\

\lipsum[16]

%----------------------------------------------------------------------------------------
%	AGRADECIMENTOS
%----------------------------------------------------------------------------------------
\section*{Agradecimentos}

\lipsum[17]

\cite{bender-koller-2020-climbing,propor-2024-international,ws-2017-brazilian}

%----------------------------------------------------------------------------------------
%	BIBLIOGRAFIA
%----------------------------------------------------------------------------------------
\renewcommand{\refname}{\color{vinho} Referências}
\bibliography{bibliografia}

\end{multicols}

%----------------------------------------------------------------------------------------
%	RODAPÉ
%----------------------------------------------------------------------------------------

\vspace{0.5cm}

\noindent\makebox[\linewidth]{
	\textcolor{vinho}{
    	\rule{\linewidth}{5pt}
	}
}
    
\noindent\makebox[\textwidth][c]
{
	\fcolorbox{vinho}{vinho}{
		\begin{minipage}[c]{0.135\linewidth}
        	\centering
        	\vspace{1.8cm}
        	\textcolor{rosaFundo}{ \Huge \boogie IX TILic }
            \vspace{1.65cm}
		\end{minipage}
        \hfill
    }

    \fcolorbox{cinzaFundo}{cinzaFundo}{
    	\begin{minipage}[c]{0.845\linewidth}
        
       		\centering
        	\vspace{0.7cm}
			\Large IX Workshop de Iniciação Científica em Tecnologia da Informação e da Linguagem Humana – TILic 2024\\
			\large Evento integrante do Simpósio Brasileiro em Tecnologia da Informação e da Linguagem Humana (STIL)\\
			Belém, Pará \\
            \vspace{0.75cm}

    	\end{minipage}}
}

\end{minipage}
\end{document}
%----------------------------------------------------------------------------------------
%%%%%%%%%%%%%%%%%%%%%%%%%%%%%%%%%%%%%%%%%%%%%%%%%%